% Options for packages loaded elsewhere
\PassOptionsToPackage{unicode}{hyperref}
\PassOptionsToPackage{hyphens}{url}
\PassOptionsToPackage{dvipsnames,svgnames,x11names}{xcolor}
%
\documentclass[
  a4paper,
  DIV=11,
  numbers=noendperiod]{scrartcl}

\usepackage{amsmath,amssymb}
\usepackage{iftex}
\ifPDFTeX
  \usepackage[T1]{fontenc}
  \usepackage[utf8]{inputenc}
  \usepackage{textcomp} % provide euro and other symbols
\else % if luatex or xetex
  \usepackage{unicode-math}
  \defaultfontfeatures{Scale=MatchLowercase}
  \defaultfontfeatures[\rmfamily]{Ligatures=TeX,Scale=1}
\fi
\usepackage{lmodern}
\ifPDFTeX\else  
    % xetex/luatex font selection
\fi
% Use upquote if available, for straight quotes in verbatim environments
\IfFileExists{upquote.sty}{\usepackage{upquote}}{}
\IfFileExists{microtype.sty}{% use microtype if available
  \usepackage[]{microtype}
  \UseMicrotypeSet[protrusion]{basicmath} % disable protrusion for tt fonts
}{}
\makeatletter
\@ifundefined{KOMAClassName}{% if non-KOMA class
  \IfFileExists{parskip.sty}{%
    \usepackage{parskip}
  }{% else
    \setlength{\parindent}{0pt}
    \setlength{\parskip}{6pt plus 2pt minus 1pt}}
}{% if KOMA class
  \KOMAoptions{parskip=half}}
\makeatother
\usepackage{xcolor}
\setlength{\emergencystretch}{3em} % prevent overfull lines
\setcounter{secnumdepth}{5}
% Make \paragraph and \subparagraph free-standing
\ifx\paragraph\undefined\else
  \let\oldparagraph\paragraph
  \renewcommand{\paragraph}[1]{\oldparagraph{#1}\mbox{}}
\fi
\ifx\subparagraph\undefined\else
  \let\oldsubparagraph\subparagraph
  \renewcommand{\subparagraph}[1]{\oldsubparagraph{#1}\mbox{}}
\fi

\usepackage{color}
\usepackage{fancyvrb}
\newcommand{\VerbBar}{|}
\newcommand{\VERB}{\Verb[commandchars=\\\{\}]}
\DefineVerbatimEnvironment{Highlighting}{Verbatim}{commandchars=\\\{\}}
% Add ',fontsize=\small' for more characters per line
\usepackage{framed}
\definecolor{shadecolor}{RGB}{241,243,245}
\newenvironment{Shaded}{\begin{snugshade}}{\end{snugshade}}
\newcommand{\AlertTok}[1]{\textcolor[rgb]{0.68,0.00,0.00}{#1}}
\newcommand{\AnnotationTok}[1]{\textcolor[rgb]{0.37,0.37,0.37}{#1}}
\newcommand{\AttributeTok}[1]{\textcolor[rgb]{0.40,0.45,0.13}{#1}}
\newcommand{\BaseNTok}[1]{\textcolor[rgb]{0.68,0.00,0.00}{#1}}
\newcommand{\BuiltInTok}[1]{\textcolor[rgb]{0.00,0.23,0.31}{#1}}
\newcommand{\CharTok}[1]{\textcolor[rgb]{0.13,0.47,0.30}{#1}}
\newcommand{\CommentTok}[1]{\textcolor[rgb]{0.37,0.37,0.37}{#1}}
\newcommand{\CommentVarTok}[1]{\textcolor[rgb]{0.37,0.37,0.37}{\textit{#1}}}
\newcommand{\ConstantTok}[1]{\textcolor[rgb]{0.56,0.35,0.01}{#1}}
\newcommand{\ControlFlowTok}[1]{\textcolor[rgb]{0.00,0.23,0.31}{#1}}
\newcommand{\DataTypeTok}[1]{\textcolor[rgb]{0.68,0.00,0.00}{#1}}
\newcommand{\DecValTok}[1]{\textcolor[rgb]{0.68,0.00,0.00}{#1}}
\newcommand{\DocumentationTok}[1]{\textcolor[rgb]{0.37,0.37,0.37}{\textit{#1}}}
\newcommand{\ErrorTok}[1]{\textcolor[rgb]{0.68,0.00,0.00}{#1}}
\newcommand{\ExtensionTok}[1]{\textcolor[rgb]{0.00,0.23,0.31}{#1}}
\newcommand{\FloatTok}[1]{\textcolor[rgb]{0.68,0.00,0.00}{#1}}
\newcommand{\FunctionTok}[1]{\textcolor[rgb]{0.28,0.35,0.67}{#1}}
\newcommand{\ImportTok}[1]{\textcolor[rgb]{0.00,0.46,0.62}{#1}}
\newcommand{\InformationTok}[1]{\textcolor[rgb]{0.37,0.37,0.37}{#1}}
\newcommand{\KeywordTok}[1]{\textcolor[rgb]{0.00,0.23,0.31}{#1}}
\newcommand{\NormalTok}[1]{\textcolor[rgb]{0.00,0.23,0.31}{#1}}
\newcommand{\OperatorTok}[1]{\textcolor[rgb]{0.37,0.37,0.37}{#1}}
\newcommand{\OtherTok}[1]{\textcolor[rgb]{0.00,0.23,0.31}{#1}}
\newcommand{\PreprocessorTok}[1]{\textcolor[rgb]{0.68,0.00,0.00}{#1}}
\newcommand{\RegionMarkerTok}[1]{\textcolor[rgb]{0.00,0.23,0.31}{#1}}
\newcommand{\SpecialCharTok}[1]{\textcolor[rgb]{0.37,0.37,0.37}{#1}}
\newcommand{\SpecialStringTok}[1]{\textcolor[rgb]{0.13,0.47,0.30}{#1}}
\newcommand{\StringTok}[1]{\textcolor[rgb]{0.13,0.47,0.30}{#1}}
\newcommand{\VariableTok}[1]{\textcolor[rgb]{0.07,0.07,0.07}{#1}}
\newcommand{\VerbatimStringTok}[1]{\textcolor[rgb]{0.13,0.47,0.30}{#1}}
\newcommand{\WarningTok}[1]{\textcolor[rgb]{0.37,0.37,0.37}{\textit{#1}}}

\providecommand{\tightlist}{%
  \setlength{\itemsep}{0pt}\setlength{\parskip}{0pt}}\usepackage{longtable,booktabs,array}
\usepackage{calc} % for calculating minipage widths
% Correct order of tables after \paragraph or \subparagraph
\usepackage{etoolbox}
\makeatletter
\patchcmd\longtable{\par}{\if@noskipsec\mbox{}\fi\par}{}{}
\makeatother
% Allow footnotes in longtable head/foot
\IfFileExists{footnotehyper.sty}{\usepackage{footnotehyper}}{\usepackage{footnote}}
\makesavenoteenv{longtable}
\usepackage{graphicx}
\makeatletter
\def\maxwidth{\ifdim\Gin@nat@width>\linewidth\linewidth\else\Gin@nat@width\fi}
\def\maxheight{\ifdim\Gin@nat@height>\textheight\textheight\else\Gin@nat@height\fi}
\makeatother
% Scale images if necessary, so that they will not overflow the page
% margins by default, and it is still possible to overwrite the defaults
% using explicit options in \includegraphics[width, height, ...]{}
\setkeys{Gin}{width=\maxwidth,height=\maxheight,keepaspectratio}
% Set default figure placement to htbp
\makeatletter
\def\fps@figure{htbp}
\makeatother
\newlength{\cslhangindent}
\setlength{\cslhangindent}{1.5em}
\newlength{\csllabelwidth}
\setlength{\csllabelwidth}{3em}
\newlength{\cslentryspacingunit} % times entry-spacing
\setlength{\cslentryspacingunit}{\parskip}
\newenvironment{CSLReferences}[2] % #1 hanging-ident, #2 entry spacing
 {% don't indent paragraphs
  \setlength{\parindent}{0pt}
  % turn on hanging indent if param 1 is 1
  \ifodd #1
  \let\oldpar\par
  \def\par{\hangindent=\cslhangindent\oldpar}
  \fi
  % set entry spacing
  \setlength{\parskip}{#2\cslentryspacingunit}
 }%
 {}
\usepackage{calc}
\newcommand{\CSLBlock}[1]{#1\hfill\break}
\newcommand{\CSLLeftMargin}[1]{\parbox[t]{\csllabelwidth}{#1}}
\newcommand{\CSLRightInline}[1]{\parbox[t]{\linewidth - \csllabelwidth}{#1}\break}
\newcommand{\CSLIndent}[1]{\hspace{\cslhangindent}#1}

\KOMAoption{captions}{tableheading}
\makeatletter
\makeatother
\makeatletter
\makeatother
\makeatletter
\@ifpackageloaded{caption}{}{\usepackage{caption}}
\AtBeginDocument{%
\ifdefined\contentsname
  \renewcommand*\contentsname{Table of contents}
\else
  \newcommand\contentsname{Table of contents}
\fi
\ifdefined\listfigurename
  \renewcommand*\listfigurename{List of Figures}
\else
  \newcommand\listfigurename{List of Figures}
\fi
\ifdefined\listtablename
  \renewcommand*\listtablename{List of Tables}
\else
  \newcommand\listtablename{List of Tables}
\fi
\ifdefined\figurename
  \renewcommand*\figurename{Figure}
\else
  \newcommand\figurename{Figure}
\fi
\ifdefined\tablename
  \renewcommand*\tablename{Table}
\else
  \newcommand\tablename{Table}
\fi
}
\@ifpackageloaded{float}{}{\usepackage{float}}
\floatstyle{ruled}
\@ifundefined{c@chapter}{\newfloat{codelisting}{h}{lop}}{\newfloat{codelisting}{h}{lop}[chapter]}
\floatname{codelisting}{Listing}
\newcommand*\listoflistings{\listof{codelisting}{List of Listings}}
\makeatother
\makeatletter
\@ifpackageloaded{caption}{}{\usepackage{caption}}
\@ifpackageloaded{subcaption}{}{\usepackage{subcaption}}
\makeatother
\makeatletter
\@ifpackageloaded{tcolorbox}{}{\usepackage[skins,breakable]{tcolorbox}}
\makeatother
\makeatletter
\@ifundefined{shadecolor}{\definecolor{shadecolor}{rgb}{.97, .97, .97}}
\makeatother
\makeatletter
\makeatother
\makeatletter
\makeatother
\ifLuaTeX
  \usepackage{selnolig}  % disable illegal ligatures
\fi
\IfFileExists{bookmark.sty}{\usepackage{bookmark}}{\usepackage{hyperref}}
\IfFileExists{xurl.sty}{\usepackage{xurl}}{} % add URL line breaks if available
\urlstyle{same} % disable monospaced font for URLs
\hypersetup{
  pdftitle={green\_paper},
  pdfauthor={Krisna Gupta},
  colorlinks=true,
  linkcolor={blue},
  filecolor={Maroon},
  citecolor={Blue},
  urlcolor={Blue},
  pdfcreator={LaTeX via pandoc}}

\title{green\_paper}
\author{Krisna Gupta}
\date{November 1, 2023}

\begin{document}
\maketitle
\begin{abstract}
awokawokawo
\end{abstract}
\ifdefined\Shaded\renewenvironment{Shaded}{\begin{tcolorbox}[interior hidden, sharp corners, frame hidden, boxrule=0pt, borderline west={3pt}{0pt}{shadecolor}, breakable, enhanced]}{\end{tcolorbox}}\fi

\hypertarget{introduction}{%
\subsection{Introduction}\label{introduction}}

Some shit from Sabzevar et al. (2017) and He, Dou, and Zhang (2017) on
calculation of cap n trade.

\hypertarget{literature-review}{%
\subsection{Literature Review}\label{literature-review}}

kwoakoawkoakwokawo

\hypertarget{method}{%
\subsection{Method}\label{method}}

Let \(Q\) be a quantity produced by the Indonesian economy which is
nested with a leontief production function with energy. That is,
\(Q=\min \{F(.),\Omega\}\), where \(F\) is a combination of factors such
as capital and labour. Let \(\Omega\) be the total energy required to
produce \(Q\) in one period. The economy produces \(\omega\) with a
fully substitute sources:

\[
\omega=w_a+w_b+w_g
\]

where \(w_a\) is the amount of clean energy used, while \(w_b\) and
\(w_g\) are coal and gas respectively. if \(p\) is a vector of prices of
the three sources of energy and \(w \in \{w_a,w_b,w_g\}\), producers in
the economy are faced with a cost minimization problem to produce \(Q\),
and by extension, \(\omega\).

\[
\begin{aligned}
\min_{w} \ &  p \cdot w \\
\mbox{subject to } \ & \omega=w_a+w_b+w_g
\end{aligned}
\]

In this setting, \(\omega\) is taken as exogenous as the consequence of
the Leontief production nest. That is, factor of production is the
driver of \(Q\) and consequently \(\omega\). This assumption allows the
use of the cost minimization technique and observe the cost impact of
idiosyncratic shock to prices.

We improve this setting by adding emission constraints. We limit total
emission coming from the use of each source of energy. Next, we limit
how much the total combination of emissions from these sources is
allowed. This variable, then, can be set exogenously to reflect the
government's preference of emission.

Let \(a,b,g\) be parameters which reflect emission generated per
megawatt hour by \(w_a, w_b, w_g\) respectively. Let \(\varepsilon\) be
the total emission generated by the Indonesian electricity sector, Then
the total emission generated by these sources is:

\[
aw_a+bw_b+gw_g=\varepsilon
\]

With the above emission constraint, we have a complete linear system as
follows:

\[
\begin{aligned}
\min_{W} \ &  p \cdot w \\
\mbox{subject to } \ & w_a+ w_b+ w_G \ge \omega \\
 & aw_a+bw_b+gw_g \le \varepsilon \\
 & w_a,w_b,w_g \ge 0\\
\end{aligned}
\]

The shock of the model can come from two exogenous variables \(p\) which
reflects a carbon tax, or \(\varepsilon\) which reflects how much carbon
quota is given in the economy as a whole.

The next step is to find a representative parameter. PLN (2021) is the
main source of \(\omega\) and \(p \cdot w\). Perusahaan Listrk Negara
(PLN) statistics is reliable since it is the sole distributor of
electricity in Indonesia. According to PLN (2021), Indonesia generates
279,511.24 Gigawatt hour (GWh) in 2021. From those, around 60\% are
produced using coal as its main source and around 23\% by some mixes of
fossil fuels. Only 17\% is generated by renewables, mostly hydroelectric
(Lolla 2021; PLN 2021).

PLN (2021) also contains data on prices per Kilowatt hour of electricity
based on sources. Total emission generated by the electricity sector is
calculated based on the emission factor and how much energy source is
used by the sector. Lastly, emission factor \(a,b,g\) are calibrated
from Febijanto (2010) and Steen (2021). The number of emission factor
varies between countries and in different reports, and emission factor
in this paper tries to balance those differences\footnote{See appendix
  for a more complete codes and parameterisation used in this paper.}.

Also cekidot
https://www.cnbcindonesia.com/news/20220119103739-4-308598/pajak-karbon-pltu-berlaku-april-2022-picu-tarif-listrik-naik

https://publications.jrc.ec.europa.eu/repository/handle/JRC21207

https://ebtke.esdm.go.id/post/2023/02/01/3414/rencana.pengembangan.pembangkit.nasional.beri.porsi.ebt.lebih.besar?lang=id

\hypertarget{results-and-discussions}{%
\subsection{Results and discussions}\label{results-and-discussions}}

The linear setting in the previous section is trivial enough to be
solved by linear programming method in Scipy (Sargent n.d.). Four
different cases are considered in this paper.

\begin{longtable}[]{@{}
  >{\raggedright\arraybackslash}p{(\columnwidth - 4\tabcolsep) * \real{0.1429}}
  >{\raggedright\arraybackslash}p{(\columnwidth - 4\tabcolsep) * \real{0.2381}}
  >{\raggedright\arraybackslash}p{(\columnwidth - 4\tabcolsep) * \real{0.6190}}@{}}
\toprule\noalign{}
\begin{minipage}[b]{\linewidth}\raggedright
case
\end{minipage} & \begin{minipage}[b]{\linewidth}\raggedright
description
\end{minipage} & \begin{minipage}[b]{\linewidth}\raggedright
model setting
\end{minipage} \\
\midrule\noalign{}
\endhead
\bottomrule\noalign{}
\endlastfoot
1 & Status quo & current share of energy use \\
2 & carbon tax & current share but carbon is taxed \\
3 & long-run, no target & long run changes if emission target \\
4 & long-run with aggresive target & like case 3 but with emission
reduced by 27\% \\
\end{longtable}

\begin{Shaded}
\begin{Highlighting}[]
\ImportTok{import}\NormalTok{ pandas }\ImportTok{as}\NormalTok{ pd}
\ImportTok{import}\NormalTok{ numpy }\ImportTok{as}\NormalTok{ np}
\ImportTok{from}\NormalTok{ scipy.optimize }\ImportTok{import}\NormalTok{ linprog}

\KeywordTok{class}\NormalTok{ carbon:}
  \CommentTok{r"""}
\CommentTok{  implements the perfect substition of energy usage with prices and emission as the contraints. Note: the default emission is calculated by share of electricity generated times its emission factor.}
\CommentTok{  """}
  
  \KeywordTok{def} \FunctionTok{\_\_init\_\_}\NormalTok{(}\VariableTok{self}\NormalTok{, omega}\OperatorTok{=}\FloatTok{279511240.0}\NormalTok{,  }\CommentTok{\#MWh}
\NormalTok{                     e}\OperatorTok{=}\FloatTok{217459744720.0}\NormalTok{,   }\CommentTok{\#KgCO2}
\NormalTok{                     pa}\OperatorTok{=}\DecValTok{1284440}\NormalTok{,         }\CommentTok{\#Rp/MWh}
\NormalTok{                     pb}\OperatorTok{=}\DecValTok{667880}\NormalTok{,          }\CommentTok{\#Rp/MWh}
\NormalTok{                     pg}\OperatorTok{=}\DecValTok{1247930}\NormalTok{,         }\CommentTok{\#Rp/MWh}
\NormalTok{                     a}\OperatorTok{=}\DecValTok{100}\NormalTok{,              }\CommentTok{\#KgCO2/MWh}
\NormalTok{                     b}\OperatorTok{=}\DecValTok{1000}\NormalTok{,             }\CommentTok{\#KgCO2/MWh}
\NormalTok{                     g}\OperatorTok{=}\DecValTok{700}\NormalTok{,              }\CommentTok{\#KgCO2/MWh}
\NormalTok{                     ba}\OperatorTok{=}\NormalTok{(}\DecValTok{0}\NormalTok{,}\VariableTok{None}\NormalTok{),}
\NormalTok{                     bb}\OperatorTok{=}\NormalTok{(}\DecValTok{0}\NormalTok{,}\VariableTok{None}\NormalTok{),}
\NormalTok{                     bg}\OperatorTok{=}\NormalTok{(}\DecValTok{0}\NormalTok{,}\VariableTok{None}\NormalTok{)}
\NormalTok{                     ):}
    \VariableTok{self}\NormalTok{.omega,}\VariableTok{self}\NormalTok{.e,}\VariableTok{self}\NormalTok{.pa,}\VariableTok{self}\NormalTok{.pb,}\VariableTok{self}\NormalTok{.pg,}\VariableTok{self}\NormalTok{.a,}\VariableTok{self}\NormalTok{.b,}\VariableTok{self}\NormalTok{.g}\OperatorTok{=}\NormalTok{omega,e,pa,pb,pg,a,b,g}
    \VariableTok{self}\NormalTok{.ba,}\VariableTok{self}\NormalTok{.bb,}\VariableTok{self}\NormalTok{.bg}\OperatorTok{=}\NormalTok{ba,bb,bg}
    
  \KeywordTok{def}\NormalTok{ hasil(}\VariableTok{self}\NormalTok{):}
\NormalTok{    omega,e,pa,pb,pg,a,b,g}\OperatorTok{=}\VariableTok{self}\NormalTok{.omega,}\VariableTok{self}\NormalTok{.e,}\VariableTok{self}\NormalTok{.pa,}\VariableTok{self}\NormalTok{.pb,}\VariableTok{self}\NormalTok{.pg,}\VariableTok{self}\NormalTok{.a,}\VariableTok{self}\NormalTok{.b,}\VariableTok{self}\NormalTok{.g}
\NormalTok{    ba,bb,bg}\OperatorTok{=}\VariableTok{self}\NormalTok{.ba,}\VariableTok{self}\NormalTok{.bb,}\VariableTok{self}\NormalTok{.bg}
    \CommentTok{\# Construct parameters}
\NormalTok{    c\_ex1 }\OperatorTok{=}\NormalTok{ np.array([pa,pb,pg])}

    \CommentTok{\# Inequality constraints}
\NormalTok{    A\_ex1 }\OperatorTok{=}\NormalTok{ np.array([[}\OperatorTok{{-}}\DecValTok{1}\NormalTok{, }\OperatorTok{{-}}\DecValTok{1}\NormalTok{,}\OperatorTok{{-}}\DecValTok{1}\NormalTok{],}
\NormalTok{                  [a,b,g]])}
\NormalTok{    b\_ex1 }\OperatorTok{=}\NormalTok{ np.array([}\OperatorTok{{-}}\NormalTok{omega,e])}

\NormalTok{    bounds\_ex2 }\OperatorTok{=}\NormalTok{ [ba,}
\NormalTok{                  bb,}
\NormalTok{                  bg]}

    \CommentTok{\# Solve the problem}
    \CommentTok{\# we put a negative sign on the objective as linprog does minimization}
\NormalTok{    res\_ex1 }\OperatorTok{=}\NormalTok{ linprog(c\_ex1, A\_ub}\OperatorTok{=}\NormalTok{A\_ex1, b\_ub}\OperatorTok{=}\NormalTok{b\_ex1,bounds}\OperatorTok{=}\NormalTok{bounds\_ex2)}
    \ControlFlowTok{return}\NormalTok{ res\_ex1}
  
  \KeywordTok{def}\NormalTok{ biaya(}\VariableTok{self}\NormalTok{):}
\NormalTok{    omega,e,pa,pb,pg,a,b,g}\OperatorTok{=}\VariableTok{self}\NormalTok{.omega,}\VariableTok{self}\NormalTok{.e,}\VariableTok{self}\NormalTok{.pa,}\VariableTok{self}\NormalTok{.pb,}\VariableTok{self}\NormalTok{.pg,}\VariableTok{self}\NormalTok{.a,}\VariableTok{self}\NormalTok{.b,}\VariableTok{self}\NormalTok{.g}
\NormalTok{    h}\OperatorTok{=}\VariableTok{self}\NormalTok{.hasil()[}\StringTok{\textquotesingle{}x\textquotesingle{}}\NormalTok{]}
\NormalTok{    v}\OperatorTok{=}\NormalTok{np.array((pa,pb,pg))}
    \ControlFlowTok{return} \BuiltInTok{print}\NormalTok{(}\SpecialStringTok{f\textquotesingle{}total biaya pembangkit listrik adalah }\SpecialCharTok{\{}\NormalTok{h }\OperatorTok{@}\NormalTok{ v}\SpecialCharTok{\}}\SpecialStringTok{ rupiah\textquotesingle{}}\NormalTok{)}
  
  \KeywordTok{def}\NormalTok{ emisi(}\VariableTok{self}\NormalTok{):}
\NormalTok{    omega,e,pa,pb,pg,a,b,g}\OperatorTok{=}\VariableTok{self}\NormalTok{.omega,}\VariableTok{self}\NormalTok{.e,}\VariableTok{self}\NormalTok{.pa,}\VariableTok{self}\NormalTok{.pb,}\VariableTok{self}\NormalTok{.pg,}\VariableTok{self}\NormalTok{.a,}\VariableTok{self}\NormalTok{.b,}\VariableTok{self}\NormalTok{.g}
\NormalTok{    h}\OperatorTok{=}\VariableTok{self}\NormalTok{.hasil()[}\StringTok{\textquotesingle{}x\textquotesingle{}}\NormalTok{]}
\NormalTok{    v}\OperatorTok{=}\NormalTok{np.array((a,b,g))}
    \ControlFlowTok{return} \BuiltInTok{print}\NormalTok{(}\SpecialStringTok{f\textquotesingle{}total emisi adalah }\SpecialCharTok{\{}\NormalTok{h }\OperatorTok{@}\NormalTok{ v}\SpecialCharTok{\}}\SpecialStringTok{ kgCO2\textquotesingle{}}\NormalTok{)}
  
  \KeywordTok{def}\NormalTok{ energi(}\VariableTok{self}\NormalTok{):}
\NormalTok{    omega,e,pa,pb,pg,a,b,g}\OperatorTok{=}\VariableTok{self}\NormalTok{.omega,}\VariableTok{self}\NormalTok{.e,}\VariableTok{self}\NormalTok{.pa,}\VariableTok{self}\NormalTok{.pb,}\VariableTok{self}\NormalTok{.pg,}\VariableTok{self}\NormalTok{.a,}\VariableTok{self}\NormalTok{.b,}\VariableTok{self}\NormalTok{.g}
\NormalTok{    sumber}\OperatorTok{=}\NormalTok{(}\StringTok{\textquotesingle{}EBT\textquotesingle{}}\NormalTok{,}\StringTok{\textquotesingle{}batubara\textquotesingle{}}\NormalTok{,}\StringTok{\textquotesingle{}fosil lain\textquotesingle{}}\NormalTok{)}
\NormalTok{    itung}\OperatorTok{=}\VariableTok{self}\NormalTok{.hasil()[}\StringTok{\textquotesingle{}x\textquotesingle{}}\NormalTok{]}
    \ControlFlowTok{for}\NormalTok{ i,j }\KeywordTok{in} \BuiltInTok{zip}\NormalTok{(sumber,itung):}
      \BuiltInTok{print}\NormalTok{(}\SpecialStringTok{f\textquotesingle{}jumlah energi dari }\SpecialCharTok{\{}\NormalTok{i}\SpecialCharTok{\}}\SpecialStringTok{ adalah }\SpecialCharTok{\{}\NormalTok{j}\SpecialCharTok{\}}\SpecialStringTok{ MWh\textquotesingle{}}\NormalTok{)}
\end{Highlighting}
\end{Shaded}

\hypertarget{reserves}{%
\subsection{Reserves}\label{reserves}}

\begin{Shaded}
\begin{Highlighting}[]
\ImportTok{import}\NormalTok{ pandas }\ImportTok{as}\NormalTok{ pd}
\ImportTok{import}\NormalTok{ numpy }\ImportTok{as}\NormalTok{ np}
\ImportTok{from}\NormalTok{ scipy.optimize }\ImportTok{import}\NormalTok{ linprog}

\NormalTok{omega}\OperatorTok{=}\FloatTok{279511240.0} \CommentTok{\#MWh}
\NormalTok{E}\OperatorTok{=}\FloatTok{619280000000.0} \CommentTok{\#KgCO2}
\NormalTok{ba}\OperatorTok{=}\NormalTok{(}\DecValTok{0}\NormalTok{,}\VariableTok{None}\NormalTok{)}
\NormalTok{bb}\OperatorTok{=}\NormalTok{(}\DecValTok{0}\NormalTok{,}\VariableTok{None}\NormalTok{)}
\CommentTok{\# Construct parameters}
\NormalTok{c\_ex1 }\OperatorTok{=}\NormalTok{ np.array([}\DecValTok{1284440}\NormalTok{, }\DecValTok{667880}\NormalTok{])}

\CommentTok{\# Inequality constraints}
\NormalTok{A\_ex1 }\OperatorTok{=}\NormalTok{ np.array([[}\OperatorTok{{-}}\DecValTok{1}\NormalTok{, }\OperatorTok{{-}}\DecValTok{1}\NormalTok{],}
\NormalTok{                  [}\DecValTok{0}\NormalTok{,}\DecValTok{350}\NormalTok{]])}
\NormalTok{b\_ex1 }\OperatorTok{=}\NormalTok{ np.array([}\OperatorTok{{-}}\NormalTok{omega,E])}

\NormalTok{bounds\_ex2 }\OperatorTok{=}\NormalTok{ [ba,}
\NormalTok{              bb]}

\CommentTok{\# Solve the problem}
\CommentTok{\# we put a negative sign on the objective as linprog does minimization}
\NormalTok{res\_ex1 }\OperatorTok{=}\NormalTok{ linprog(c\_ex1, A\_ub}\OperatorTok{=}\NormalTok{A\_ex1, b\_ub}\OperatorTok{=}\NormalTok{b\_ex1,bounds}\OperatorTok{=}\NormalTok{bounds\_ex2)}

\NormalTok{res\_ex1}
\end{Highlighting}
\end{Shaded}

\begin{verbatim}
        message: Optimization terminated successfully. (HiGHS Status 7: Optimal)
        success: True
         status: 0
            fun: 186679966971200.0
              x: [ 0.000e+00  2.795e+08]
            nit: 1
          lower:  residual: [ 0.000e+00  2.795e+08]
                 marginals: [ 6.166e+05  0.000e+00]
          upper:  residual: [       inf        inf]
                 marginals: [ 0.000e+00  0.000e+00]
          eqlin:  residual: []
                 marginals: []
        ineqlin:  residual: [ 0.000e+00  5.215e+11]
                 marginals: [-6.679e+05 -0.000e+00]
 mip_node_count: 0
 mip_dual_bound: 0.0
        mip_gap: 0.0
\end{verbatim}

\hypertarget{section}{%
\subsection{}\label{section}}

\hypertarget{bibliography}{%
\subsection*{Bibliography}\label{bibliography}}
\addcontentsline{toc}{subsection}{Bibliography}

\hypertarget{refs}{}
\begin{CSLReferences}{1}{0}
\leavevmode\vadjust pre{\hypertarget{ref-febijanto10}{}}%
Febijanto, Irhan. 2010. {``Perhitungan Faktor Emisi Di Sistem Jaringan
Ketenagalistrikan Jawa-Madura-Bali.''} \emph{Jurnal Teknologi Lingkungan
BPPT} 11 (2): 227--37. \url{https://doi.org/10.29122/jtl.v11i2.1207}.

\leavevmode\vadjust pre{\hypertarget{ref-HDZ17}{}}%
He, Ping, Guowei Dou, and Wei Zhang. 2017. {``Optimal Production
Planning and Cap Setting Under Cap-and-Trade Regulation.''} Journal
Article. \emph{The Journal of the Operational Research Society} 68 (9):
1094--1105.
https://doi.org/\url{https://doi.org/10.1057/s41274-016-0123-1}.

\leavevmode\vadjust pre{\hypertarget{ref-ember}{}}%
Lolla, Muyi, Aditya AND Yang. 2021. {``Indonesia: Indonesia Defies
Global Trend with More Coal in the Generation-Mix.''} ember-climate;
Global Electricity Review 2021: G20 profile.

\leavevmode\vadjust pre{\hypertarget{ref-pln}{}}%
PLN. 2021. {``Statistik PLN.''} 01001-220630. Perusahaan Listrik Negara.

\leavevmode\vadjust pre{\hypertarget{ref-SEBK17}{}}%
Sabzevar, Nikoo, S. T. Enns, Joule Bergerson, and Janne Kettunen. 2017.
{``Modeling Competitive Firms' Performance Under Price-Sensitive Demand
and Cap-and-Trade Emissions Constraints.''} Journal Article.
\emph{International Journal of Production Economics} 184: 193--209.
https://doi.org/\url{https://doi.org/10.1016/j.ijpe.2016.10.024}.

\leavevmode\vadjust pre{\hypertarget{ref-quant}{}}%
Sargent, John, Thomas J. AND Stachurski. n.d. {``Intermediate
Quantitative Economics with Python.''} quantecon.org.

\leavevmode\vadjust pre{\hypertarget{ref-jrc}{}}%
Steen, M. 2021. {``Greenhouse Gas Emissions from Fossil Fuel Fired Power
Generation Systems.''} JRC21207. JRC Publications Repository.
\url{https://publications.jrc.ec.europa.eu/repository/handle/JRC21207}.

\end{CSLReferences}



\end{document}
